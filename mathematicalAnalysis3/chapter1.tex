\chapter{实数系与复数系}
\chapter*{引导}为啥引入实数
\section{例子}方程 \[ p^2 = 2 \] 没有有理数解。因为有理数可以\( p=m/n \)这种方式表示,其中m及n都是整数,而且可以选的不全是偶数。如果有p满足方程,那么\(m^2 = 2 * n^2\)。因此\(m^2\)是偶数,m是偶数,\(m^2\)是4的倍数,因此\(n^2\)也是偶数,n也就是偶数。这与之前的选择矛盾。
\section{评注}
\section{定义} 集合
\section{定义} 在第一章,Q表示所有有理数的集合。
\subparagraph*{}
\chapter*{有序集}
\section{定义} 序
\section{定义} 有序集
\section{定义} 上界,下界。上有界,下有界
\section{定义} 最小上界/上确届,最大下界/下确界
\section{例子} 
\section{定义} 最小上界性
\section{定理} 最小上界性集S的子集E,$E \subset S$,E不空,且E下有界,则$ _{inf}E \in S$
\subparagraph*{}
\chapter*{域}
\section{定义} 域
\section{评注} 一些简写
\section{命题} 加法公理蕴含的一些陈述
\section{命题} 乘法法公理蕴含的一些陈述
\section{命题} 域公理蕴含的一些陈述
\section{定义} 有序域
\section{命题} 有序域公理蕴含的一些陈述
\subparagraph*{}
\chapter*{实数域}
\section{定理}有最小上界性的有序域R存在。同时R可以包含着Q作为其子域。
\paragraph{}R的元叫做实数。
\subparagraph{}首先R是定义的。是在有理数上添加了无限和任意的概念扩展来。证明里用到了有理数,使用了无限/任意的概念,而实际上无限位不循环的有理数,就不是有理数。因为添加的是无限和任意的概念,因此实数不仅包含代数数,也包含超越数。
\section{定理}
\paragraph{}(a)如果 \(x \in R\),\(y \in R\)且$ x > 0 $,那么必定存在正整数n,使得 $$ nx > y $$
\paragraph{}(b)如果 \(x \in R\),\(y \in R\)且$ x < y $,那么必定存在$p \in Q$,合于 $$ x < p < y $$
\paragraph{}(a)的证明使用了R的最小上界性。令A为$\{ nx| x\in$正整数$\}$的集,$ A\subset R $,所以可以让 $ \alpha = _{sup}A $。又因为$x > 0$,这样$\alpha - x < \alpha$,$\alpha -x$就不是A的上界,于是至少存在某个正整数m使得$mx > \alpha-x \Rightarrow (m+1)x > \alpha$,而m+1是正整数,这就与$\alpha$是上确界矛盾了。
\subparagraph{}这里实际上有个推论,因为1>0,所以对于任意指定的$ x \in R $,都有一个正整数n使得$n>x$。
\section{定义} 实数$x > 0$,整数$n > 0$,有且只有一个实数y使得$y > 0$且$y^n = x$。
\section*{推论}
\section{十进小数}
\subparagraph*{}
\chapter*{广义实数系}
\section{定义}引入了$ +\infty, -\infty $
\subparagraph*{}
\chapter*{复数域}
\section{定义}复数
x=(a, b) y=(c, d)
x+y=(a + c, b + d)
xy=(ac-bd, ad + bc)
\section{定理}复数满足域公理
\section{定理}实数是复数的子域
\section{定义} $i = (0, 1) $
\section{定理} $i^2 = -1 $
\section{定理} a,b是实数,$(a, b) = a + bi$
\section{定义} 共轭数,实部和虚部
\section{定理} 共轭相关的一些简单算式
\section{定义} 复数的绝对值。$ | z | = (z\overline{z})^{\frac{1}{2}} $
\section{定理} 复数绝对值的一些定理。
\section{记号} 求和记法 $$ x_1 + x_2 + ... + x_n = \sum_{j=1}^{n}x_j $$
\section{定理} Schwarz不等式。
如果$ a_1,a_2,...,a_n $及$ b_1,b_2,...,b_n $都是复数,那么 $$ |\sum_{j=1}^{n}(a_j\overline{b}_j)|^2 \leq \sum_{j=1}^{n}|a_j|^2 \sum_{j=1}^{n}|b_j|^2$$
\subparagraph*{}
\chapter*{欧氏空间}
\section{定义}正整数k,$ R^k $
\section{定义} $ x,y,z \in R^k $,而a是实数,那么有些算式。
\section{评注} $ R^k $可以看作是第二章的度量空间。$ R^1 $和$ R^2 $
\subparagraph*{}
\chapter*{附录}
\subparagraph*{}
\chapter*{习题}除了明确说明,这里提到的数都是实数
\subsection*{1}如果$ r(r \ne 0) $是有理数,而x是无理数,证明r+x及rx是无理数
\paragraph{证明}反证法很容易。r+x及rx都是实数。如果是有理数,会推论出x是有理数的结论。
\subsection*{2} 不存在平方为12的有理数。
\paragraph{证明}类似平方为2的证明,会导致不同为偶数的p、q都是偶数的结论。
\subsection*{3} 命题1.15
\paragraph{证明}
\subparagraph{} (a) $x\ne 0$,并且xy = xz $ \Rightarrow \frac{1}{x}xy=\frac{1}{x}xz \Rightarrow 1y=1z \Rightarrow y=z$
\subparagraph{} (b) $x\ne 0$,并且xy = x $ \Rightarrow xy = x1 \Rightarrow y = 1 $
\subparagraph{} (c) $x\ne 0$,并且xy = 1 $ \Rightarrow xy = x\frac{1}{x} \Rightarrow y = \frac{1}{x} $
\subparagraph{} (d) $x\ne 0$,由(c)有$\frac{1}{x}x=1 \Rightarrow x=\frac{1}{\frac{1}{x}}$
\subsection*{4} E是某有序集的非空子集,$\alpha$是E的下界,而$\beta$是E的上界,证明$\alpha \le \beta$
\paragraph{证明}令$x \in E$,则$ \alpha \le x $且$x \le \beta$,因此$\alpha \le \beta$
\subsection*{5} 设A是非空实数集,下有界。令-A是所有-x的集,$x\in A$,求证$$_{inf}A=-_{sup}(-A)$$
\paragraph{证明} 令$a=_{inf}A$,则$a\le x \Rightarrow -a \ge -x$,因此-a是-A的上界。如果$-b < -a \Rightarrow b > a $,b不是A的下界,于是有$x < b, x\in A \Rightarrow -b < -x $即-b不是-A的上界,因此-a是-A的最小上界,即$_{inf}A=-_{sup}(-A)$
\subsection*{6}
\paragraph{证明}
\subsection*{7}
\paragraph{证明}
\subsection*{8} 证明复数域中不能定义顺序关系以使其成为有序域。
\paragraph{证明}$(0,1),(0, -1),(1, 0)和(-1, 0)$在反复平方后都会到达(1,0)。
\subparagraph{}无论$(0, 1) < 0$还是$(0, 1) > 0$,都会有$(-1, 0) > 0$,那么$(-1, 0)^2=(1, 0) > 0$,但是这会产生矛盾$ 0 = (-1, 0) + (1, 0) > 0$.
\subsection*{9}
\paragraph{证明}
\subsection*{10}
\paragraph{证明}
\subsection*{11}
\paragraph{证明}
\subsection*{12}
\paragraph{证明}
\subsection*{13}
\paragraph{证明}
\subsection*{14}
\paragraph{证明}
\subsection*{15}
\paragraph{证明}
\subsection*{16}
\paragraph{证明}
\subsection*{17}
\paragraph{证明}
\subsection*{18}
\paragraph{证明}
\subsection*{19}
\paragraph{证明}
\subsection*{20}
\paragraph{证明}
